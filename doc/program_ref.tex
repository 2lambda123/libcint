\documentclass{article}
\usepackage{amsmath}
\usepackage[mathletters]{ucs}
\usepackage[utf8x]{inputenc}
\usepackage[breaklinks=true,unicode=true]{hyperref}
\setlength{\parindent}{0pt}
\setlength{\parskip}{6pt plus 2pt minus 1pt}
\setcounter{secnumdepth}{0}

\title{Program Reference}

\begin{document}
\maketitle

\tableofcontents

\section{Interface}

\subsection{C routine}

\begin{verbatim}
f(buf, shls, atm, natm, bas, nbas, env);
\end{verbatim}
\begin{itemize}
\item
  buf: column-major double precision array.
  \begin{itemize}
  \item
    for 1e integrals of shells (i,j), data are stored as
    \verb![i1j1 i2j1 ... ]!
  \item
    for 2e integrals of shells (i,j\textbar{}k,l), data are stored as\\
    \verb![i1j1k1l1 i2j1k1l1 ... i1j2k1l1 ... i1j1k2l1 ... ]!
  \item
    complex data are stored as two double elements, first is real,
    followed by imaginary, e.g.~[Re Im Re Im \ldots{}]
  \end{itemize}
\item
  shls: 0-based basis/shell indices.
  \begin{itemize}
  \item
    int[2] for 1e integrals
  \item
    int[4] for 2e integrals
  \end{itemize}
\item
  atm: int[natm*6], list of atoms. For ith atom, the 6 slots of
  atm[i] are
  \begin{itemize}
  \item
    \verb!atm[i*6+0]! nuclear charge of atom i
  \item
    \verb!atm[i*6+1]! env offset to save coordinates
    (\verb!env[atm[i*6+1]]!, \verb!env[atm[i*6+1]+1]!,
    \verb!env[atm[i*6+1]+2]!) are (x,y,z)
  \item
    \verb!atm[i*6+2]! nuclear model of atom i, = 2 indicates gaussian
    nuclear model
  \item
    \verb!atm[i*6+3]! unused
  \item
    \verb!atm[i*6+4]! unused
  \item
    \verb!atm[i*6+5]! unused
  \end{itemize}
\item
  natm: int, number of atoms, natm has no effect
  \textbf{except nuclear attraction} integrals
\item
  bas: int[nbas*8], list of basis. For ith basis, the 8 slots of
  bas[i] are
  \begin{itemize}
  \item
    \verb!bas[i*8+0]! 0-based index of corresponding atom
  \item
    \verb!bas[i*8+1]! angular momentum
  \item
    \verb!bas[i*8+2]! number of primitive GTO in basis i
  \item
    \verb!bas[i*8+3]! number of contracted GTO in basis i
  \item
    \verb!bas[i*8+4]! kappa for spinor GTO.\\ \textless{} 0 the basis
    \ensuremath{\sim} j = l + 1/2.\\ \textgreater{} 0 the basis
    \ensuremath{\sim} j = l - 1/2.\\ = 0 the basis includes both j = l
    + 1/2 and j = l - 1/2
  \item
    \verb!bas[i*8+5]! env offset to save exponents of primitive GTOs.
    e.g.~10 exponents \verb!env[bas[i*8+5]]! \ldots{}
    \verb!env[bas[i*8+5]+9]!
  \item
    \verb!bas[i*8+6]! env offset to save column-major contraction
    coefficients. e.g.~10 primitive -\textgreater{} 5 contraction needs
    a $10\times 5$ array
  \end{itemize}
\end{itemize}
\begin{verbatim}
env[bas[i*8+6]  ] | env[bas[i*8+6]+10] |     | env[bas[i*8+6]+40]
env[bas[i*8+6]+1] | env[bas[i*8+6]+11] |     | env[bas[i*8+6]+41]
 .                |  .                 | ... |  .                
 .                |  .                 |     |  .                
env[bas[i*8+6]+9] | env[bas[i*8+6]+19] |     | env[bas[i*8+6]+49]
\end{verbatim}
\begin{verbatim}
- `bas[i*8+7]` unused
\end{verbatim}
\begin{itemize}
\item
  nbas: int, number of bases, nbas has no effect
\item
  env: double[], save the value of coordinates, exponents,
  contraction coefficients
\end{itemize}
\subsection{Fortran routine}

\begin{verbatim}
call f(buf, shls, atm, natm, bas, nbas, env)
\end{verbatim}
\begin{itemize}
\item
  atm and bas are 2D integer array
  \begin{itemize}
  \item
    atm(1:6,i) is the (charge, offset\_coord, nuclear\_model, unused,
    unused, unused) of the ith atom
  \item
    bas(1:8,i) is the (atom\_index, angular, num\_primitive\_GTO,
    num\_contract\_GTO, kappa, offset\_exponent, offset\_coeff, unused)
    of the ith basis
  \end{itemize}
\item
  parameters are the same to the C function. Note those offsets
  atm(2,i) bas(6,i) bas(7,i) are 0-based.
\item
  buf is 2D/4D double precision/double complex array
\end{itemize}
\subsection{Supported angular momentum}

\begin{itemize}
\item
  1e integrals $l_{max}$ = 6
\item
  2e integrals $l_{max}$ = 4
\end{itemize}
\subsection{Data ordering}

\begin{itemize}
\item
  for Cartesian GTO, the output data in buf are sorted
  as\\\begin{tabular}{l|l|l|l}
\hline
    s shell & p shell & d shell & ... \\
\hline
    ...     & ...     & ...     & \\
    s       & p $x$   & d $xx$  & \\
    s       & p $y$   & d $xy$  & \\
    ...     & p $z$   & d $xz$  & \\
            & p $x$   & d $yy$  & \\
            & p $y$   & d $yz$  & \\
            & p $z$   & d $zz$  & \\
            & ...     & ...     & \\
\hline
\end{tabular}
\item
  for real spheric GTO, the output data in buf are sorted
  as\\\begin{tabular}{l|l|l|l|l}
\hline
    s shell & p shell & d shell     & f shell         & ... \\
\hline
    ...     & ...     & ...         & ...             & \\
    s       & p $y$   & d $xy     $ & f $y(3x^2-y^2)$ & \\
    s       & p $z$   & d $yz     $ & f $xyz        $ & \\
    ...     & p $x$   & d $z^2    $ & f $yz^2       $ & \\
            & p $y$   & d $xz     $ & f $z^3        $ & \\
            & p $z$   & d $x^2-y^2$ & f $xz^2       $ & \\
            & p $x$   & ...         & f $z(x^2-y^2) $ & \\
            & ...     &             & f $x(x^2-3y^2)$ & \\
            &         &             & ...             & \\
\hline
\end{tabular}
\item
  for spinor GTO, the output data in buf correspond
  to\\\begin{tabular}{l|l|l|l|l}
\hline
    ... & kappa=0,p shell & kappa=1,p shell & kappa=0,d shell & ... \\
\hline
        & ...             & ...             & ...             &     \\
        & $p_{1/2}(-1/2)$ & $p_{1/2}(-1/2)$ & $d_{3/2}(-3/2)$ &     \\
        & $p_{1/2}( 1/2)$ & $p_{1/2}( 1/2)$ & $d_{3/2}(-1/2)$ &     \\
        & $p_{3/2}(-3/2)$ & $p_{1/2}(-1/2)$ & $d_{3/2}( 1/2)$ &     \\
        & $p_{3/2}(-1/2)$ & $p_{1/2}( 1/2)$ & $d_{3/2}( 3/2)$ &     \\
        & $p_{3/2}( 1/2)$ & $p_{1/2}(-1/2)$ & $d_{5/2}(-5/2)$ &     \\
        & $p_{3/2}( 3/2)$ & $p_{1/2}( 1/2)$ & $d_{5/2}(-3/2)$ &     \\
        & $p_{1/2}(-1/2)$ & ...             & $d_{5/2}(-1/2)$ &     \\
        & $p_{1/2}( 1/2)$ &                 & $d_{3/2}(-3/2)$ &     \\
        & $p_{3/2}(-3/2)$ &                 & $d_{3/2}(-1/2)$ &     \\
        & $p_{3/2}(-1/2)$ &                 & ...             &     \\
        & ...             &                 &                 &     \\
\hline
\end{tabular}
\end{itemize}
\subsection{Tensor}

integrals like Gradients have tensor components. the output data
is

\begin{itemize}
\item
  3-component tensor
  \begin{itemize}
  \item
    X \verb!buf(:,0)!
  \item
    Y \verb!buf(:,1)!
  \item
    Z \verb!buf(:,2)!
  \end{itemize}
\item
  9-component tensor
  \begin{itemize}
  \item
    XX \verb!buf(:,0)!
  \item
    XY \verb!buf(:,1)!
  \item
    XZ \verb!buf(:,2)!
  \item
    YX \verb!buf(:,3)!
  \item
    YY \verb!buf(:,4)!
  \item
    YZ \verb!buf(:,5)!
  \item
    ZX \verb!buf(:,6)!
  \item
    ZY \verb!buf(:,7)!
  \item
    ZZ \verb!buf(:,8)!
  \end{itemize}
\end{itemize}
\section{function list}

\begin{itemize}
\item
  Cartesian GTO integrals
\item
  \verb!cgto_cart!\\
\item
  \verb!cint1e_ovlp_cart! \[\langle i| j\rangle \]
\item
  \verb!cint1e_nuc_cart! \[\langle i| V_{nuc} | j\rangle \]
\item
  \verb!cint1e_kin_cart!
  \[.5\langle i| \vec{p} \cdot \vec{p} j\rangle \]
\item
  \verb!cint1e_ia01p_cart!
  \[\langle i| \frac{\vec{r}}{r^3}| \times \vec{\nabla} j\rangle \]
\item
  \verb!cint1e_irxp_cart!
  \[\langle i| \vec{r}_c \times \vec{\nabla} j\rangle \]
\item
  \verb!cint1e_iking_cart!
  \[0.5i\langle \vec{p} \cdot \vec{p} i| U_gj\rangle \]
\item
  \verb!cint1e_iovlpg_cart! \[i \langle i| U_gj\rangle \]
\item
  \verb!cint1e_inucg_cart! \[i \langle i| V_{nuc}| U_gj\rangle \]
\item
  \verb!cint1e_ipovlp_cart! \[\langle \vec{\nabla} i|j\rangle \]
\item
  \verb!cint1e_ipkin_cart!
  \[0.5\langle \vec{\nabla} i| \vec{p} \cdot \vec{p} j\rangle \]
\item
  \verb!cint1e_ipnuc_cart!
  \[\langle \vec{\nabla} i| V_{nuc}|j\rangle \]
\item
  \verb!cint1e_iprinv_cart!
  \[\langle \vec{\nabla} i| r^{-1}|j\rangle \]
\item
  \verb!cint1e_rinv_cart! \[\langle i| r^{-1} |j\rangle \]
\item
  \verb!cint2e_cart! \[(ij|kl)\]
\item
  \verb!cint2e_ig1_cart! \[i(i U_g j|kl)\]
\item
  \verb!cint2e_ip1_cart! \[(\vec{\nabla} i j|kl)\]

\item
  Spheric GTO integrals
\item
  \verb!cgto_spheric!\\
\item
  \verb!cint1e_ovlp_sph! \[\langle  i| j\rangle \]
\item
  \verb!cint1e_nuc_sph! \[\langle  i| V_{nuc}| j\rangle \]
\item
  \verb!cint1e_kin_sph! \[0.5\langle i| \vec{p} \cdot pj\rangle \]
\item
  \verb!cint1e_ia01p_sph!
  \[\langle i| \frac{\vec{r}}{r^3}| \times \vec{\nabla} j\rangle \]
\item
  \verb!cint1e_irxp_sph!
  \[\langle i| \vec{r}_c \times \vec{\nabla} j\rangle \]
\item
  \verb!cint1e_iking_sph!
  \[0.5i\langle \vec{p} \cdot \vec{p} i| U_gj\rangle \]
\item
  \verb!cint1e_iovlpg_sph! \[i\langle i| U_gj\rangle \]
\item
  \verb!cint1e_inucg_sph! \[i\langle i| V_{nuc}| U_gj\rangle \]
\item
  \verb!cint1e_ipovlp_sph! \[\langle \vec{\nabla} i|j\rangle \]
\item
  \verb!cint1e_ipkin_sph!
  \[0.5\langle \vec{\nabla} i| \vec{p} \cdot pj\rangle \]
\item
  \verb!cint1e_ipnuc_sph!
  \[\langle \vec{\nabla} i| V_{nuc}|j\rangle \]
\item
  \verb!cint1e_iprinv_sph!
  \[\langle \vec{\nabla} i| r^{-1}|j\rangle \]
\item
  \verb!cint1e_rinv_sph! \[\langle i| r^{-1} |j\rangle \]
\item
  \verb!cint2e_sph! \[(ij|kl)\]
\item
  \verb!cint2e_ig1_sph! \[i(i U_g j|kl)\]
\item
  \verb!cint2e_ip1_sph! \[(\vec{\nabla} i j|kl)\]

\item
  Spinor GTO integrals
\item
  \verb!cgto_spinor!\\
\item
  \verb!cint1e_ovlp! \[\langle  i| j\rangle \]
\item
  \verb!cint1e_nuc! \[\langle  i| V_{nuc} |j\rangle \]
\item
  \verb!cint1e_nucg! \[\langle  i| V_{nuc} | U_gj\rangle \]
\item
  \verb!cint1e_srsr!
  \[\langle \vec{\sigma}\cdot\vec{r} i| \vec{\sigma}\cdot\vec{r}j\rangle \]
\item
  \verb!cint1e_sr! \[\langle \vec{\sigma}\cdot\vec{r} i|j\rangle \]
\item
  \verb!cint1e_srsp!
  \[\langle \vec{\sigma}\cdot\vec{r} i| \vec{\sigma}\cdot\vec{p}j\rangle \]
\item
  \verb!cint1e_spsp!
  \[\langle \vec{\sigma}\cdot\vec{p} i| \vec{\sigma}\cdot\vec{p}j\rangle \]
\item
  \verb!cint1e_sp! \[\langle \vec{\sigma}\cdot\vec{p} i|j\rangle \]
\item
  \verb!cint1e_spspsp!
  \[\langle \vec{\sigma}\cdot\vec{p} i| \vec{\sigma}\cdot\vec{p} \vec{\sigma}\cdot\vec{p}j\rangle \]
\item
  \verb!cint1e_spnuc!
  \[\langle \vec{\sigma}\cdot\vec{p} i| V_{nuc} |j\rangle \]
\item
  \verb!cint1e_spnucsp!
  \[\langle \vec{\sigma}\cdot\vec{p} i| V_{nuc} | \vec{\sigma}\cdot\vec{p}j\rangle \]
\item
  \verb!cint1e_srnucsr!
  \[\langle \vec{\sigma}\cdot\vec{r} i| V_{nuc} | \vec{\sigma}\cdot\vec{r}j\rangle \]
\item
  \verb!cint1e_sa10sa01!
  \[0.5\langle\vec{\sigma} \times \vec{r}_c i|\vec{\sigma} \times \frac{\vec{r}}{r^3} |j\rangle \]
\item
  \verb!cint1e_ovlpg! \[\langle i|U_g j\rangle \]
\item
  \verb!cint1e_sa10sp!
  \[0.5\langle\vec{r}_c \times\vec{\sigma} i| \vec{\sigma}\cdot\vec{p}j\rangle \]
\item
  \verb!cint1e_sa10nucsp!
  \[0.5\langle\vec{r}_c \times\vec{\sigma} i| V_{nuc} | \vec{\sigma}\cdot\vec{p}j\rangle \]
\item
  \verb!cint1e_sa01sp!
  \[\langle i| \frac{\vec{r}}{r^3} \times\vec{\sigma} | \vec{\sigma}\cdot\vec{p}j\rangle \]
\item
  \verb!cint1e_spgsp!
  \[\langle U_g \vec{\sigma}\cdot\vec{p} i| \vec{\sigma}\cdot\vec{p}j\rangle \]
\item
  \verb!cint1e_spgnucsp!
  \[\langle U_g \vec{\sigma}\cdot\vec{p} i| V_{nuc} | \vec{\sigma}\cdot\vec{p}j\rangle \]
\item
  \verb!cint1e_spgsa01!
  \[\langle U_g \vec{\sigma}\cdot\vec{p} i| \frac{\vec{r}}{r^3} \times\vec{\sigma} |j\rangle \]
\item
  \verb!cint1e_ipovlp! \[\langle \vec{\nabla} i|j\rangle \]
\item
  \verb!cint1e_ipkin!
  \[0.5\langle \vec{\nabla} i| p \cdot pj\rangle \]
\item
  \verb!cint1e_ipnuc! \[\langle \vec{\nabla} i| V_{nuc}|j\rangle \]
\item
  \verb!cint1e_iprinv! \[\langle \vec{\nabla} i| r^{-1}|j\rangle \]
\item
  \verb!cint1e_ipspnucsp!
  \[\langle \vec{\nabla} \vec{\sigma}\cdot\vec{p} i| V_{nuc}| \vec{\sigma}\cdot\vec{p}j\rangle \]
\item
  \verb!cint1e_ipsprinvsp!
  \[\langle \vec{\nabla} \vec{\sigma}\cdot\vec{p} i| r^{-1}| \vec{\sigma}\cdot\vec{p}j\rangle \]
\item
  \verb!cint2e! \[(ij|kl)\]
\item
  \verb!cint2e_spsp1!
  \[(\vec{\sigma}\cdot\vec{p} i \vec{\sigma}\cdot\vec{p} j| k l)\]
\item
  \verb!cint2e_spsp1spsp2!
  \[(\vec{\sigma}\cdot\vec{p} i \vec{\sigma}\cdot\vec{p} j| \vec{\sigma}\cdot\vec{p} k \vec{\sigma}\cdot\vec{p} l)\]
\item
  \verb!cint2e_srsr1!
  \[(\vec{\sigma}\cdot\vec{r} i \vec{\sigma}\cdot\vec{r} j| kl)\]
\item
  \verb!cint2e_srsr1srsr2!
  \[(\vec{\sigma}\cdot\vec{r} i \vec{\sigma}\cdot\vec{r} j| \vec{\sigma}\cdot\vec{r} k \vec{\sigma}\cdot\vec{r}l)\]
\item
  \verb!cint2e_sa10sp1!
  \[0.5 (\vec{r}_c \times\vec{\sigma} i \vec{\sigma}\cdot\vec{p} j| kl)\]
\item
  \verb!cint2e_sa10sp1spsp2!
  \[0.5 (\vec{r}_c \times\vec{\sigma} i \vec{\sigma}\cdot\vec{p} j| \vec{\sigma}\cdot\vec{p} k \vec{\sigma}\cdot\vec{p} l)\]
\item
  \verb!cint2e_g1! \[(i U_g j| kl)\]
\item
  \verb!cint2e_spgsp1!
  \[(\vec{\sigma}\cdot\vec{p} i U_g \vec{\sigma}\cdot\vec{p} j| kl)\]
\item
  \verb!cint2e_g1spsp2!
  \[(i U_g j| \vec{\sigma}\cdot\vec{p} k \vec{\sigma}\cdot\vec{p}l)\]
\item
  \verb!cint2e_spgsp1spsp2!
  \[(\vec{\sigma}\cdot\vec{p} i U_g \vec{\sigma}\cdot\vec{p} j| \vec{\sigma}\cdot\vec{p} k \vec{\sigma}\cdot\vec{p}l)\]
\item
  \verb!cint2e_ip1! \[(\vec{\nabla} i j| kl)\]
\item
  \verb!cint2e_ipspsp1!
  \[(\vec{\nabla} \vec{\sigma}\cdot\vec{p} i \vec{\sigma}\cdot\vec{p} j| kl)\]
\item
  \verb!cint2e_ip1spsp2!
  \[(\vec{\nabla} i j| \vec{\sigma}\cdot\vec{p} k \vec{\sigma}\cdot\vec{p}l)\]
\item
  \verb!cint2e_ipspsp1spsp2!
  \[(\vec{\nabla} \vec{\sigma}\cdot\vec{p} i \vec{\sigma}\cdot\vec{p} j| \vec{\sigma}\cdot\vec{p} k \vec{\sigma}\cdot\vec{p}l)\]
\item
  \verb!cint2e_ssp1ssp2!
  \[( i \vec{\sigma}\vec{\sigma}\cdot\vec{p} j|k \vec{\sigma}\vec{\sigma}\cdot\vec{p}l)\]

\end{itemize}

\end{document}

